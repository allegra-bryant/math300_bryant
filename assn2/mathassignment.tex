\documentclass{article}
\usepackage[utf8]{inputenc}
\usepackage{fullpage}
\usepackage{amsmath}
\usepackage{amssymb}
\usepackage{amsthm}
\pagestyle{empty}

\begin{document}
\newtheorem{theorem}{Theorem}[section]
\theoremstyle{definition}
\theoremstyle{remark}
\newtheorem{definition}{Definition}[section]

\noindent
of B in terms of A's entries, we group the four equations we obtain from this relation as follows:

\begin{equation}
\left \{\begin{array}{cc}
a_{11}b_{11} + a_{12}b_{21} & = 1,  \\
a_{21}b_{11} + a_{22}b_{21} & = 0,
\end {array}
\right
\text{and} \left \{\begin{array}{cc}
a_{11}b_{12} + a_{12}b_{22} & = 0,  \\
a_{21}b_{12} + a_{22}b_{22} & = 1,
\end {array}
\right
\end{equation}

\noindent
The above systems are simultaneous equations in the unknowns $(b_{11}, b_{21})$ and $(b_{12},b_{22})$, respectively. Simple manipulations lead to


\begin{equation}
b_{11} = \frac{a_{22}}{D}, \ \
b_{12} = -\frac{a_{12}}{D}, \ \
b_{21} = -\frac{a_{21}}{D}, \ \
b_{22} = \frac{a_{11}}{D} \ \
\end{equation}
 

\noindent
where D, called the \textit{determinant} of the matrix A, is
\begin{equation}
    D = a_{11}a_{22} - a_{12}a_{21}.
\end{equation}
\noindent
Hence, 

\begin{equation}
    A^{-1}=B=\frac{1}{a_{11}a_{22}-a_{12}a_{21}}
\left[
\begin {array}{cc}
a_{22}& -a_{12}\\
-a_{21}& a_{11}
\end{array}
\right].
\end{equation}

\noindent
Clearly if the determinant of the matrix A is zero, the formulas in (4) are not valid and A will not have an inverse. Such matrices are called \textit{singular} and will play a significant role when we discuss eigenvalues and eigenvectors. By contrast \textit{nonsingular} or \textit{invertible} matrices, are those with nonzero determinants, will have unique inverses, which are in turn used to determine the unique solution to the system of algebraic equations.

\begin{equation}
    A\boldsymbol{x}=\boldsymbol{b}.
\end{equation}

\noindent
To see this, multiply both sides of the above equation by $A^{-1}$ to get

\begin{displaymath}
A^{-1}(A\boldsymbol{x})=A^{-1}\boldsymbol{b}.
\end{displaymath}

\noindent
Since $A^{-1}A = I$, the left side reduces to $\boldsymbol{x}$ and we end up with

\begin{equation}
    \boldsymbol{x}=A^{-1}\boldsymbol{b}
\end{equation}

\noindent
as the unique solution to (5). What we have illustrated is important enough that we state it as a theorem. 

\begin{theorem} \textbf{(Existence and Uniqueness of Solutions)}
\end{theorem}

\noindent 
\textit{Consider a system of linear algebraic equations in the form of (5). Then (5) has the unique solution (6) if and only if A is nonsingular.}  \\

We have only illustrated this theorem in the contact of 2 by 2 matrices. It turns out that its statement and conclusion are valid for \textit{n x n} matrices, a claim that we can easily verify once we generalize the concept

\end{document}
